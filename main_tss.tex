\documentclass[
11pt, 
a4paper,
DIV = 14,
twoside,
twocolumn, % z w e i s e i t i g e r A u f b a u
parskip =half, % H a l b e Z e i l e A b s a t z a b s t a n d
headsepline, % K o p f z e i l e n s e p a r a t i o n s l i n i e
openright, % N e u e s K a p i t e l b e g i n n t r e c h t s
]{scrreprt}

\usepackage{lipsum}
\usepackage{graphicx}
%\usepackage[chapter]{placeins}
\usepackage{fontenc}
\usepackage{scrlayer-scrpage}
\usepackage[ngerman]{babel}
\usepackage{lmodern}
\usepackage{blindtext}
\usepackage[utf8]{inputenc}
\usepackage[backend=biber,style=alphabetic]{biblatex}
\usepackage{csquotes}
\addbibresource{ref/ref.bib}


\begin{document}
	\ohead{\includegraphics[height=45pt]{bilder/thi.png.png}}
	
	%%%%%%%%% KOPF %%%%%%%%%%%%%%%%%%%%%%%%%%
	\title{Welche Rolle spielen künstliche Intelligenz und Machine Learning bei der Optimierung von Energieversorgungssystemen?}
	\author{Nico Elsner\\
		Studiengang: Robotik}
	\date{\today}
	\maketitle
	\tableofcontents
	\thispagestyle{empty}
	
	\chapter{Einleitung}
	\setcounter{page}{1}
	\blindtext
	\section{Hintergrund und Motivation des Themas}
	\blindtext
	\section{Zielsetzung und Fragestellung der Arbeit}
	\blindtext


	\chapter{Grundlagen der künstlichen Intelligenz und Machine Learning}
	\blindtext
	\section{Definition und Abgrenzung}
	\blindtext
	\section{Typen von Machine Learning Algorithmen}
	\blindtext


	\chapter{Optimierung von Energieversorgungssystemen}
	\blindtext
	\section{Überblick über Energieversorgungssystemen}
	\blindtext
	\section{Herausforderungen bei der Optimierung}
	\blindtext
	\section{Allgemeine Möglichkeiten der Optimierung}
	\blindtext
	

	\chapter{Rolle von künstlicher Intelligenz und Machine Learning bei der Optimierung von Energieversorgungssystemen}
	\blindtext
	\section{Potenzial und Vorteile von künstlicher Intelligenz und Machine Learning}
	\blindtext
	\section{Beispiele und Anwendungen von künstlicher Intelligenz und Machine Learning in der Energieversorgung}
	\blindtext


	\chapter{Kritische Betrachtung und Ausblick}
	\blindtext
	\section{Grenzen und Risiken von künstlicher Intelligenz}
	\blindtext
	\section{Perspektiven}
	\blindtext


	\chapter{Fazit und Ausblick}
	\blindtext

\end{document}