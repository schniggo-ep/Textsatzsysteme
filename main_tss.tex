\documentclass[conference]{IEEEtran}
\IEEEoverridecommandlockouts
%immer rechts kapitel beginnen

%Default-einstellungen für das Dokument

\usepackage[T1]{fontenc}    %Schriftcodierung
\usepackage[utf8]{inputenc}    %Dateicodierung
\usepackage[ngerman]{babel} %Sprachpaket(Silbentrennung,überschriften, Abbildungen)
\usepackage{graphicx}
\usepackage{lmodern} %Vektorschrift
\usepackage{sansmath}
\usepackage{blindtext}
\usepackage{svg}
\usepackage{amsfonts}
\usepackage{amsmath}
\usepackage{svg}
\usepackage{caption}
\captionsetup{
	figurename=Abb:,
}
\usepackage[
	automark, %automatische Kolumne
]{scrlayer-scrpage}
\pagestyle{scrheadings}
\clearscrheadfoot


%Literatur anpassen
\usepackage[
	backend=biber,
	style=numeric-verb,
	sorting=none % numeric-verb, numeric-comb
]{biblatex}
\addbibresource{ref.bib}

%Schriftanpassungen
\renewcommand{\familydefault}{\sfdefault} %serifenlose Schrift
\sansmath			%Aktivieren von Paket "Sansmath"

\ofoot[\pagemark]{\pagemark} %Seitenzahl in Fußzeile außen

\graphicspath{{bilder/}{}}  %Pfad zu den Bildern. mehrere pfade mit {{}{}{}} 


\title{Welche Rolle spielen künstliche Intelligenz und Machine Learning bei der Optimierung von Energieversorgungssystemen?}
\author{
	\IEEEauthorblockN{Nico Elsner}
	\IEEEauthorblockA{
		\textit{Technische Hochschule Ingolstadt} \\
		16. Juni 2023 %\Abgabe
	}
}


\begin{document}
%\ohead{\includegraphics[height=45pt]{bilder/thi.png.png}}

\maketitle
\author{{\Large Nico Elsner}}

\begin{abstract}
	Die Optimierung von Energieversorgungssystemen ist angesichts der steigenden
Nachfrage nach effizienter, zuverlässiger und nachhaltiger Energieversorgung
von großer Bedeutung. In diesem Zusammenhang spielen künstliche Intelligenz
(KI) und maschinelles Lernen (ML) eine immer wichtigere Rolle. Dieser Artikel
untersucht die Rolle von KI und ML bei der Optimierung von
Energieversorgungssystemen. Es werden verschiedene Anwendungen von KI und ML
vorgestellt, darunter Prognose und Optimierung, Lastmanagement und
Nachfragesteuerung, Anlagenüberwachung und Wartung, Integration erneuerbarer
Energien sowie Energieeffizienz und Verbraucherempfehlungen. Darüber hinaus
werden die Herausforderungen diskutiert, die mit der Optimierung von
Energieversorgungssystemen verbunden sind, wie die Komplexität der Systeme, die
Volatilität erneuerbarer Energien, die Skalierbarkeit und Flexibilität, die
Sicherheit und Zuverlässigkeit sowie die wirtschaftlichen und regulatorischen
Aspekte. Die Erkenntnisse dieser Arbeit tragen dazu bei, das Verständnis für
die Rolle von KI und ML bei der Optimierung von Energieversorgungssystemen zu
vertiefen und zeigen Potenziale sowie Herausforderungen auf, die bei der
Einführung dieser Technologien berücksichtigt werden sollten.
\end{abstract}

\begin{IEEEkeywords}
	Künstliche Intelligenz, Energieversorgungssysteme, KI, AI, Neuronale Netze, Machine Learning, Maschinelles Lernen
\end{IEEEkeywords}

%% Einleitung
\section{Einleitung}
\subsection{Hintergrund und Motivation des Themas}
Die Energieversorgungssysteme stehen heutzutage vor einer Vielzahl von
Herausforderungen. Der steigende Energiebedarf, die begrenzte Verfügbarkeit
konventioneller Energieressourcen und die zunehmende Bedeutung erneuerbarer
Energien erfordern innovative Lösungen, um die Effizienz und Nachhaltigkeit der
Energieversorgung zu verbessern. In diesem Zusammenhang spielen künstliche
Intelligenz (KI) und maschinelles Lernen (ML) eine immer wichtigere Rolle.

Die Grundidee hinter dem Einsatz von KI und ML in der Optimierung von
Energieversorgungssystemen besteht darin, dass diese Technologien in der Lage
sind, große Mengen an Daten zu analysieren, Muster zu erkennen, Vorhersagen zu
treffen und adaptive Entscheidungen zu treffen. Dadurch können
Energieversorgungssysteme effizienter gestaltet, die Zuverlässigkeit erhöht und
Kosten gesenkt werden.

Ein zentrales Anwendungsgebiet von KI und ML liegt in der Prognose von
Energieerzeugung und -nachfrage. Durch den Einsatz von ML-Algorithmen können
präzisere Vorhersagen über den zukünftigen Energiebedarf gemacht werden, was
eine optimierte Planung der Energieerzeugung und -verteilung ermöglicht.
Darüber hinaus können ML-Modelle genutzt werden, um Lastmanagement-Strategien
zu entwickeln, um den Energieverbrauch zu optimieren und die Spitzenlasten zu
reduzieren.

Ein weiteres wichtiges Einsatzgebiet von KI und ML liegt in der Optimierung von
Stromnetzen und -verteilung. Hier können intelligente Algorithmen eingesetzt
werden, um den Energiefluss zu steuern, Engpässe zu identifizieren und die
Netzstabilität zu gewährleisten. ML kann auch bei der Erkennung von Fehlern und
Störungen in Energieversorgungssystemen helfen, indem Anomalien in den Daten
erkannt und potenzielle Probleme frühzeitig gemeldet werden.

Die Integration erneuerbarer Energien stellt ebenfalls eine große
Herausforderung dar, da sie von Natur aus volatil und schwer vorhersehbar sind.
Hier kommen KI und ML zum Einsatz, um Vorhersagemodelle für die Stromerzeugung
aus erneuerbaren Quellen zu entwickeln, um die Integration in das Gesamtsystem
zu optimieren. Dies ermöglicht eine bessere Ausnutzung erneuerbarer Energien
und eine verbesserte Integration mit konventionellen Energiequellen.

Neben den technischen Aspekten gibt es auch weitere wichtige Faktoren, die bei
der Optimierung von Energieversorgungssystemen berücksichtigt werden müssen.
Datenschutz und Sicherheit spielen eine entscheidende Rolle, da die
Verarbeitung großer Mengen an Energie- und Verbrauchsdaten eine Herausforderung
für den Schutz sensibler Informationen darstellt. Darüber hinaus stellen sich
auch ethische Fragen, wie beispielsweise der faire Zugang zu Energie oder die
Vermeidung von Diskriminierung bei der Nutzung von KI-gesteuerten
Energiesystemen.

In Anbetracht all dieser Aspekte ist es von großer Bedeutung, die Rolle von KI
und ML bei der Optimierung von Energieversorgungssystemen zu untersuchen. Durch
die Nutzung des Potenzials dieser Technologien können wir eine effizientere,
nachhaltigere und zuverlässigere Energieversorgung erreichen. In dieser
Seminararbeit werden wir die verschiedenen Einsatzbereiche von KI und ML in der
Energieversorgung untersuchen und ihre Auswirkungen sowie mögliche
Herausforderungen und zukünftige Entwicklungen analysieren.
\section{Zielsetzung und Fragestellung der Arbeit}
Die Zielsetzung dieser Arbeit besteht darin, die Rolle von künstlicher
Intelligenz und maschinellem Lernen bei der Optimierung von
Energieversorgungssystemen zu untersuchen. Es wird angestrebt, ein umfassendes
Verständnis dafür zu entwickeln, wie KI und ML eingesetzt werden können, um die
Effizienz, Zuverlässigkeit und Nachhaltigkeit der Energieversorgung zu
verbessern. Dabei sollen die verschiedenen Anwendungen und Herausforderungen
von KI und ML in diesem Kontext beleuchtet werden. Des Weiteren werden mögliche
Perspektiven und zukünftige Entwicklungen aufgezeigt, um einen Ausblick auf das
Potenzial dieser Technologien für die Energieversorgungssysteme der Zukunft zu
geben. Letztendlich zielt die Arbeit darauf ab, einen Beitrag zur aktuellen
Diskussion über die Integration von KI und ML in den Bereich der
Energieversorgung zu leisten und Erkenntnisse für Entscheidungsträger, Forscher
und Praktiker bereitzustellen.
\section{Grundlagen der künstlichen Intelligenz und Machine Learning}
\input{kapitel/Grundlagen der künstlichen Intelligenz und Machine Learning.tex}
\subsection{Abgrenzung von Machine Learning zu künstlicher Intelligenz}
Künstliche Intelligenz ist ein umfassendes Konzept, das sich auf die
Entwicklung von Computern oder Maschinen bezieht, die in der Lage sind,
Aufgaben auszuführen, die normalerweise menschliche Intelligenz erfordern.
KI-Systeme sollen menschenähnliches Denken, Lernen, Problemlösen,
Entscheidungsfindung und Sprachverarbeitung nachbilden. Sie können auf
verschiedenen Ansätzen basieren, darunter maschinelles Lernen.

Maschinelles Lernen ist ein Teilgebiet der künstlichen Intelligenz, das
sich darauf konzentriert, Algorithmen und Techniken zu entwickeln, die es
Computern ermöglichen, aus Erfahrungen zu lernen und Aufgaben zu erledigen,
ohne explizit programmiert zu sein. Im Kern geht es beim maschinellen Lernen
darum, Muster und Zusammenhänge in Daten zu erkennen und darauf basierend
Vorhersagen zu treffen oder Entscheidungen zu treffen.

Der Unterschied zwischen KI und ML besteht darin, dass KI ein übergeordnetes
Konzept ist, das sich auf den allgemeinen Bereich der Entwicklung intelligenter
Systeme bezieht, während ML eine konkrete Methode ist, die von KI-Forschern und
-Entwicklern verwendet wird, um Computermodelle zu erstellen, die auf Daten
lernen können.

Ein KI-System kann auf verschiedenen Techniken basieren, einschließlich
maschinellem Lernen, aber es kann auch andere Ansätze wie logisches
Schlussfolgern, Expertensysteme, neuronale Netzwerke oder symbolische
Verarbeitung verwenden. Das maschinelle Lernen hingegen ist eine spezifische
Methode innerhalb der KI, bei der Algorithmen verwendet werden, um aus Daten zu
lernen und Vorhersagen zu treffen, ohne dass explizite Regeln programmiert
werden müssen.

Zusammenfassend lässt sich sagen, dass KI ein übergeordnetes Konzept ist, das
sich auf die Entwicklung intelligenter Systeme bezieht, während ML eine
spezifische Methode ist, die in der KI eingesetzt wird, um Computermodelle zu
erstellen, die aus Erfahrungen lernen können. Maschinelles Lernen ist ein
wichtiger Teil der KI, aber es gibt auch andere Techniken und Ansätze, die in
KI-Systemen verwendet werden können.\cite{metaversekunstliche}\cite{kirste2019einleitung}
\subsection{Neuronale Netze}
Künstliche neuronale Netze sind die Grundfunktionsweise von System, welche
künstliche Intelligenz nutzen. Diese Netze bestehen aus einer großen Menge an
Neuronen, welche, genauso wie die Neuronen im menschlichen Gehirn, eine
Vielzahl an Daten verarbeitet. In der folgenden Abbildung ist der Ablauf der
Arbeitsweise von künstlichen neuronalen Netzen exemplarisch dargestellt:

\begin{figure}[h]
    \centering
    \includesvg[width=1\columnwidth]{bilder/neuronalenetze_prinzip.svg}
    \caption[width=0.4\textwidth]{Ablauf des Trainings eines neuronalen Netzes}
    \label{fig:trainingneuronalesnetz}
\end{figure}

Das Training eines neuronalen Netzes beginnt mit dem Bereitstellen der
Informationen, welches oft Signale oder Bildpixel sind. Zum Beispiel
Handschriften, Gesichter oder Daten von Wetterberichten. Darauffolgend wird der
Aufbau des neuronalen Netzes bestimmt, hinsichtlich der Anzahl an Hidden-Layers
und Neuronen, gefolgt von der Verarbeitung der Eingangsinformationen durch
Training mit verschiedenen Lernverfahren. Das Lernverfahren, mit dem geringsten
Fehler wird ausgewählt, aus welchem dann die Antwort des neuronalen Netzes in,
als Vektor kodierter Form ausgegeben wird.\cite{dorn2016programmieren}
\subsection{Typen von Machine Learning Algorithmen}
\begin{itemize}
    \item \textbf{Unsupervised Learning}:
          Beim unüberwachten Lernen gibt es keine Zielwerte, sondern das Ziel besteht
darin, Muster und Strukturen in den Daten zu entdecken. Das Modell lernt
eigenständig, wie es die Daten gruppieren oder Zusammenhänge finden kann.
Clustering und Dimensionsreduktion sind Beispiele für unüberwachtes Lernen.
    \item \textbf{Semi-supervised Learning}:
          Beim halbüberwachten Lernen werden Modelle mit einer Kombination aus markierten
und nicht markierten Daten trainiert. Das Modell nutzt die vorhandenen
markierten Daten, um Informationen zu generalisieren und die nicht markierten
Daten zu nutzen, um weitere Muster zu erkennen.
    \item \textbf{Supervised Learning}:
          Der Grundgedanke bei supervised Learning ist, dass die Ausgabewerte bekannt
sind. Man unterscheidet hierbei zwischen konkreten Ausgabewerten, sowie
kontinuierliche Werte. Diese beiden Wertearten bestimmen, ob es sich bei dem
supervised-learning Ansatz um eine Klassifikation oder eine Regression handelt.
Bei einem kontinuierlichen Ausgangstyp ist es eine Regression, bei einem
diskreten Ausgabewert spricht man von einer Klassifikation. Diese
Unterscheidung ist nur von den Ausgabetypen abhängig und nicht von den
Eingabewerten.\cite{matzka2021unuberwachtes}\\

    \item \textbf{Reinforcement Learning}:
          Reinforcement learning unterscheidet sich grundlegend von den bisher genannten
Lernarten. Während bei un-, semisupervised learning, sowie supervised learning
immer sehr viele Eingabedaten für das Lernsystem bereitgestellt werden müssen,
generiert reinforcement learning diese selbst.Ein sogenannter Agent interagiert
hierfür in einer bestimmten Umgebung. Das Ziel des Agenten besteht darin,
Aktionen auszuführen, um eine maximale Belohnung zu erhalten. Durch
Erfahrungen, Belohnungen und Bestrafungen passt der Agent sein Verhalten an, um
optimale Entscheidungen in der Umgebung zu treffen. Der Prozess ist in
Abbildung(\ref{fig:disreinforcementlearning}) noch einmal grafisch
aufgearbeitet. Dieser Lernansatz ist perfekt für Szenarien, in denen man keine,
oder nur sehr schwierig Trainingsdaten bereitstellen kann.
\cite{lanquillon2019grundzuge}

\begin{figure}[!h]
    \centering
    \includesvg[width=0.8\columnwidth]{bilder/reinforcementlearning.svg}
    \caption[width=0.4\textwidth]{Ablauf bei Reinforcement Learning}
    \label{fig:disreinforcementlearning}
\end{figure}
    \item \textbf{Transfer Learning}:
          Transferlernen bezieht sich auf die Übertragung von Wissen oder Fähigkeiten
eines gelernten Modells auf eine neue, ähnliche Aufgabe. Das bereits trainierte
Modell wird als Ausgangspunkt genommen und auf die neue Aufgabe feinabgestimmt,
um den Trainingsprozess zu beschleunigen und die Leistung zu verbessern.
\end{itemize}

\section{Optimierung von Energieversorgungssystemen}
\input{kapitel/Optimierung von Energieversorgungssystemen.tex}
\subsection{Überblick über Energieversorgungssystemen}
\begin{itemize}
    \item \textbf{Konventionelle Energieversorgungssysteme}: 
        Konventionelle Energieversorgungssysteme basieren hauptsächlich auf fossilen
Brennstoffen wie Kohle, Erdöl und Erdgas. Diese Systeme umfassen thermische
Kraftwerke, die die Energie aus der Verbrennung von fossilen Brennstoffen
gewinnen, um elektrische Energie zu erzeugen. Sie sind weit verbreitet und
haben in der Vergangenheit den Großteil der Energieversorgung abgedeckt.    
    \item \textbf{Erneuerbare Energieversorgungssysteme}: 
        Erneuerbare Energieversorgungssysteme nutzen natürliche Ressourcen wie Sonne,
Wind, Wasser und Biomasse zur Energieerzeugung. Solarenergie wird durch
Photovoltaik- oder Solarthermieanlagen gewonnen, während Windenergie mithilfe
von Windturbinen erzeugt wird. Wasserkraftwerke nutzen die Energie des
fließenden Wassers, und Biomasseanlagen wandeln organische Materialien in
Energie um. Diese Systeme sind nachhaltiger und umweltfreundlicher, da sie
erneuerbare Ressourcen nutzen und weniger CO2-Emissionen verursachen.
    \item \textbf{Dezentrale Energieversorgungssysteme}: 
        Dezentrale Energieversorgungssysteme zeichnen sich durch eine lokalisierte
Energieerzeugung und -verteilung aus. Anstatt auf zentrale Kraftwerke
angewiesen zu sein, werden kleine Energieerzeugungsanlagen wie Solaranlagen auf
Dächern von Gebäuden oder kleine Windturbinen eingesetzt. Diese Systeme bieten
die Möglichkeit, Energie vor Ort zu erzeugen und den Energieverlust durch den
Transport über weite Strecken zu reduzieren.
    \item \textbf{Intelligente Energieversorgungssysteme}: 
        Da in den vergangenen Jahrzehnten die Anzahl an erneuerbaren
Energieversorgungssystemen extrem gestiegen ist, und auch in Zukunft aufgrund
des Klimwandels immer mehr werden wird, werden Möglichkeiten benötigt, die
Energie bestens zu verteilen. Eine Herausforderung hierbei ist die Inkontinenz
von Energieerzeugung von z.B. Windkraftanlagen begründet durch die schwankende
Windstärke. Intelligente Energieversorgungssysteme nutzen fortschrittliche
Technologien wie intelligente Netze (Smart Grids), um die Effizienz,
Zuverlässigkeit und Integration erneuerbarer Energien zu verbessern. Smart
Grids ermöglichen eine bidirektionale Kommunikation zwischen Energieerzeugern,
Verbrauchern und dem Netz, wodurch ein optimierter Energiefluss, eine bessere
Laststeuerung und die Integration von dezentralen Erzeugungsanlagen möglich
sind.\cite{lund2012electricity} Durch den Einsatz von Sensoren, Datenanalyse und Steuerungssystemen
können intelligente Energieversorgungssysteme den Energieverbrauch und die
Nachfrage besser prognostizieren und optimieren.
\end{itemize}
\subsection{Herausforderungen bei der Optimierung}
Die Optimierung von Energieversorgungssystemen steht vor einer Reihe von
Herausforderungen, die es zu bewältigen gilt. Eine der zentralen
Herausforderungen besteht in der Komplexität dieser Systeme.
Energieversorgungssysteme bestehen aus einer Vielzahl von Komponenten und
Akteuren, wie beispielsweise Energieerzeugung, -übertragung, -verteilung und
-verbrauch. Die Koordination und Optimierung all dieser Komponenten erfordert
eine umfassende Analyse und Steuerung, um effiziente Ergebnisse zu erzielen.

Ein weiterer wichtiger Aspekt ist die Volatilität erneuerbarer Energien. Die
Integration von erneuerbaren Energien wie Solarenergie und Windenergie in das
Energieversorgungssystem stellt eine Herausforderung dar. Diese Energiequellen
unterliegen starken Schwankungen aufgrund von Wetterbedingungen und weisen
volatilere Erzeugungsmuster auf als konventionelle Energieressourcen.

Darüber hinaus stehen Energieversorgungssysteme vor Herausforderungen im
Zusammenhang mit der Skalierbarkeit und Flexibilität. Da die Energienachfrage
schwankt und sich die Energieerzeugungsmuster ändern, müssen die Systeme in der
Lage sein, sich an diese Veränderungen anzupassen und eine zuverlässige
Energieversorgung sicherzustellen.

Die Sicherheit und Zuverlässigkeit des Energieversorgungssystems ist eine
weitere Herausforderung. Da das Energienetz ein attraktives Ziel für
Cyberangriffe darstellt, müssen Maßnahmen ergriffen werden, um die Systeme vor
potenziellen Bedrohungen zu schützen und die Verfügbarkeit der
Energieversorgung zu gewährleisten.

Nicht zuletzt sind auch wirtschaftliche und regulatorische Herausforderungen zu
berücksichtigen. Die Optimierung von Energieversorgungssystemen erfordert
Investitionen in Infrastruktur, Technologien und Forschung. Zudem müssen
regulatorische Rahmenbedingungen geschaffen werden, die die Integration neuer
Technologien und Geschäftsmodelle ermöglichen und den Übergang zu einer
nachhaltigen Energieversorgung unterstützen.

Insgesamt stehen die Optimierung von Energieversorgungssystemen vor
verschiedenen Herausforderungen, die von der Komplexität der Systeme über die
Integration erneuerbarer Energien bis hin zu Fragen der Sicherheit,
Skalierbarkeit und Wirtschaftlichkeit reichen. Die Bewältigung dieser
Herausforderungen erfordert einen interdisziplinären Ansatz, der
technologische, wirtschaftliche, regulatorische und soziale Aspekte
berücksichtigt.

\subsection{Allgemeine Möglichkeiten der Optimierung}
Die Optimierung der Energiesysteme ist in unterschiedliche Bereiche aufteilbar.
Zum einen sollte auf die Performance der Systeme geachtet werden. Durch den
Mehrverbrauch an Energie im Zuge der Digitalisierung und des
Bevölkerungswachstums, ist es von immer größerer Bedeutung, ein sicheres,
starkes Energienetz zur Verfügung zu haben. !!!!!GRAFIK EINBINDEN
ENERGIEVERBRAUCH!!! Die Steigerung der Performance kann durch den Ausbau von
bereits existierenden Anlagen geschehen, jedoch sollte man auch versuchen, in
der Forschung neue Möglichkeiten zu finden, den Wirkungsgrad von Systemen zu
erhöhen. Dadurch könnte man dem hohen Flächenverbrauch, durch Solar- und
Windparks entgegenwirken.

Ein weiterer Bereich, auf welchen geachtet werden muss, ist die
Umweltverträglichkeit der Energieversorgungssysteme. Vor dem Hintergrund des
Klimawandels, sollte möglichst vermieden werden, dass große Mengen an $CO_2$ und
anderen schädlichen Gasen, durch konventionelle Energieversorgungssysteme in die Atmosphäre
gelangen. Durch Verbesserung von regenerativen Systemen kann diesem Vorgang vorgebeugt werden.

\section{Rolle von künstlicher Intelligenz und Machine Learning bei der Optimierung von Energieversorgungssystemen}
\input{kapitel/Rolle von künstlicher Intelligenz und Machine Learning bei der Optimierung von Energieversorgungssystemen.tex}
\subsection{Potenzial und Vorteile von künstlicher Intelligenz und Machine Learning}
Künstliche Intelligenz (KI) und maschinelles Lernen (ML) spielen eine immer
wichtigere Rolle bei der Optimierung von Energieversorgungssystemen. Diese
Technologien bieten innovative Ansätze zur Verbesserung der Effizienz,
Zuverlässigkeit und Nachhaltigkeit der Energieerzeugung, -verteilung und
-nutzung. In diesem Kontext können KI und ML verschiedene Hauptaufgaben
übernehmen.

Ein wichtiger Aspekt ist die Prognose und Optimierung von Energieverbrauch und
-erzeugung. Aufgrund der neuartigen Systeme und der diskontinuierlich
Energieerzeugung von erneuerbarer Energie ist es wichtig, bereits heute 
abzuschätzen, was morgen an Energie verbraucht und erzeugt wird. 
Man unterscheidet zwischen zwei den verschiedenen Arten von Prognosen:

\begin{itemize}
    \item \textbf{Verbrauchsprognose}:
          Bei den Verbrauchern wird an dem Anschlusspunkt an das öffentliche
Versorgungsnetz gemessen. Hier gibt es eine sogenannte registrierte
Leistungsmessung im Viertelstundenraster, welche bei Verbrauchern angewandt
wird, welche mehr als 100.000 kWh/a benötigt. Bei der anderen Messung werden
Gewerbe- und Haushaltskunden betrachtet, welche unter 100.000 kWh/a
verbrauchen.\cite{deppekunstliche} Diese Messungen dienen als Eingaben für die
künstliche Intelligenz, welche daraus Muster ergeben und folglich präzise
Prognosen geben kann.
    \item \textbf{Erzeugungsprognose}:
          Schwieriger ist die Prognose der Erzeugung von Energie, da diese, durch den
immer größeren Einfluss von fluktuierenden Energieversorgungssysten wie
Windkraftanlagen beachtet werden muss. Die Vorhersage ist zudem für
Netzbetreiber von besonderer Bedeutung, da diese dazu verpflichtet sind, ein
kontinuierliches Energieangebot zu bieten. Sollten Sie dies nicht durch eigene
Energieversorgungssysteme ermöglichen können, muss der Netzbetreiber von
anderen Anbietern Energie zukaufen. \cite{deppekunstliche}
\end{itemize}

Durch die Analyse
historischer Daten können Modelle entwickelt werden, um den zukünftigen
Energiebedarf vorherzusagen. Dies ermöglicht eine verbesserte Planung der
Energieerzeugung und -verteilung, um Engpässe zu vermeiden und die Effizienz zu
maximieren. Zudem können ML-Algorithmen zur Lösung komplexer
Optimierungsprobleme eingesetzt werden, beispielsweise zur Optimierung von
Energieflüssen in intelligenten Netzwerken oder zur Ressourcenoptimierung in
dezentralen Energieversorgungssystemen.

Eine weitere Rolle von KI und ML liegt im Lastmanagement und der
Nachfragesteuerung. Durch die Analyse von Echtzeitdaten können Modelle
entwickelt werden, um den Energiebedarf in verschiedenen Szenarien
vorherzusagen und entsprechende Maßnahmen zur Laststeuerung zu empfehlen. Dies
kann die Netzstabilität verbessern und den Bedarf an zusätzlicher
Energieerzeugung während Spitzenzeiten reduzieren.

Darüber hinaus können KI und ML zur Überwachung von Energieerzeugungsanlagen
und zur frühzeitigen Erkennung von Wartungsbedarf eingesetzt werden. Durch die
Analyse von Sensordaten und anderen Betriebsparametern können Modelle
entwickelt werden, um Anomalien und potenzielle Ausfälle vorherzusagen. Dies
ermöglicht eine proaktive Instandhaltung und verringert die Ausfallzeiten von
Anlagen, was zu einer verbesserten Effizienz und Verfügbarkeit führt.

Die Integration erneuerbarer Energien stellt eine weitere Herausforderung dar,
die durch KI und ML bewältigt werden kann. Durch die Analyse von Wetterdaten,
Netzwerkleistung und historischen Mustern können Modelle entwickelt werden, um
die Vorhersage und Integration erneuerbarer Energien zu verbessern.

Schließlich können KI und ML Verbrauchern dabei helfen, ihren Energieverbrauch
zu optimieren und energieeffiziente Entscheidungen zu treffen. Durch die
Analyse von Verbrauchsdaten und Verhaltensmustern können personalisierte
Empfehlungen gegeben werden, beispielsweise der Einsatz von energieeffizienten
Geräten oder die Anpassung des Nutzungsverhaltens.

Insgesamt spielen KI und ML eine entscheidende Rolle bei der Optimierung von
Energieversorgungssystemen, indem sie zu einer effizienteren Nutzung von
Energie, einer besseren Integration erneuerbarer Energien und einer
verbesserten Steuerung von Lasten beitragen. Diese Technologien tragen dazu
bei, die Herausforderungen im Energiesektor anzugehen und den Übergang zu einer
nachhaltigeren und effizienteren Energieversorgung zu unterstützen.
\subsection{Beispiele und Anwendungen von künstlicher Intelligenz und Machine Learning in der Energieversorgung}
In der Energieversorgung bieten künstliche Intelligenz (KI) und maschinelles
Lernen (ML) eine Vielzahl von Anwendungen und Beispielen, die zur Optimierung
und Effizienzsteigerung beitragen. Eine wichtige Anwendung besteht in der
Prognose und Optimierung von Energieverbrauch und -erzeugung. Durch die Analyse
historischer Daten und die Anwendung von ML-Algorithmen können Modelle
entwickelt werden, die den zukünftigen Energiebedarf prognostizieren und die
optimale Nutzung von Energiequellen ermöglichen. Dies hilft
Energieversorgungsunternehmen, die Energieerzeugung effizienter zu planen,
Engpässe zu vermeiden und die Kosten zu senken.

Ein weiteres Beispiel ist das Lastmanagement und die Nachfragesteuerung. KI und
ML ermöglichen eine präzise Vorhersage des Energiebedarfs in Echtzeit.
Basierend auf diesen Vorhersagen können intelligente Systeme empfehlen, wann
und wie die Energieverteilung optimiert werden sollte, um Spitzenlasten zu
reduzieren und die Netzstabilität zu gewährleisten. Durch die Implementierung
von Lastmanagementstrategien können Energieversorger die Effizienz verbessern,
Kosten senken und die Nachfragesteuerung erleichtern.

Des Weiteren können KI und ML bei der Überwachung und Wartung von
Energieerzeugungsanlagen eingesetzt werden. Durch die kontinuierliche Analyse
von Sensordaten und Betriebsparametern können Modelle entwickelt werden, um
Anomalien und potenzielle Ausfälle frühzeitig zu erkennen. Dies ermöglicht eine
proaktive Instandhaltung und reduziert die Ausfallzeiten der Anlagen. Durch die
Implementierung von Predictive Maintenance können Energieversorger die
Betriebseffizienz steigern, die Wartungskosten senken und die Zuverlässigkeit
der Energieerzeugung verbessern.

Die Integration erneuerbarer Energien stellt ebenfalls eine wichtige Anwendung
von KI und ML dar. Durch die Analyse von Wetterdaten, Netzwerkleistung und
historischen Mustern können Modelle entwickelt werden, um die Vorhersage und
Integration erneuerbarer Energien zu verbessern. KI-gesteuerte Systeme können
den Energiefluss optimieren, um die Schwankungen in der erneuerbaren
Energieerzeugung auszugleichen und eine stabile und zuverlässige
Energieversorgung sicherzustellen.

Ein weiteres Beispiel ist die Nutzung von KI und ML zur Förderung der
Energieeffizienz und zur Bereitstellung von Verbraucherempfehlungen. Durch die
Analyse von Verbrauchsdaten und Verhaltensmustern können personalisierte
Empfehlungen gegeben werden, wie beispielsweise der Einsatz von
energieeffizienten Geräten, die Optimierung der Zeitpläne für den
Energieverbrauch oder die Identifizierung von Energieeinsparungspotenzialen.
Diese personalisierten Empfehlungen helfen den Verbrauchern, ihren
Energieverbrauch zu optimieren und energieeffiziente Entscheidungen zu treffen.

Insgesamt bieten KI und ML in der Energieversorgung zahlreiche Anwendungen und
Beispiele, die zur Effizienzsteigerung, Kostenoptimierung, Integration
erneuerbarer Energien und Verbesserung des Verbraucherverhaltens beitragen. Der
Einsatz dieser Technologien ermöglicht es Energieversorgungsunternehmen, sich
den Herausforderungen der modernen Energiewelt anzupassen und eine
nachhaltigere und effizientere Energieversorgung zu erreichen.

\section{Kritische Betrachtung und Ausblick}
\input{kapitel/Kritische Betrachtung und Ausblick.tex}
\subsection{Grenzen und Risiken von künstlicher Intelligenz}
Der Einsatz künstlicher Intelligenz (KI) in Energieversorgungssystemen birgt
sowohl Grenzen als auch Risiken, die berücksichtigt werden müssen. Eine der
Grenzen liegt in der Qualität und Verfügbarkeit von Daten. KI-Modelle erfordern
hochwertige und ausreichende Daten, um zuverlässige Vorhersagen und
Entscheidungen treffen zu können. In der Energieversorgung können jedoch
Engpässe bei der Datenqualität und -verfügbarkeit auftreten, insbesondere wenn
es um spezifische Daten zu erneuerbaren Energien oder Verbrauchsverhalten geht.
Eine unzureichende Datenbasis kann die Genauigkeit und Leistungsfähigkeit von
KI-Anwendungen beeinträchtigen.

Ein weiterer Aspekt sind die ethischen Überlegungen. KI-Systeme können
Entscheidungen treffen, die weitreichende Auswirkungen auf die
Energieversorgung und die Gesellschaft haben. Es besteht das Risiko von
Vorurteilen, Diskriminierung oder unfairen Entscheidungen, wenn die Algorithmen
nicht angemessen trainiert oder kalibriert sind. Es ist wichtig,
sicherzustellen, dass KI-Systeme fair und transparent arbeiten, um potenzielle
negative Auswirkungen zu vermeiden.

Darüber hinaus besteht das Risiko von Fehlinterpretationen oder
Fehlentscheidungen durch KI-Modelle. Obwohl KI-Systeme in der Lage sind, Muster
und Zusammenhänge in den Daten zu erkennen, sind sie nicht immun gegen Fehler.
Wenn die Trainingsdaten nicht repräsentativ oder unvollständig sind, können die
Vorhersagen und Empfehlungen der KI-Systeme ungenau oder fehlerhaft sein. Eine
sorgfältige Validierung und Überwachung der KI-Modelle ist daher von
entscheidender Bedeutung, um mögliche Risiken zu minimieren.

Des Weiteren können Sicherheitsrisiken auftreten. KI-Systeme, die mit
Energieversorgungssystemen verbunden sind, könnten anfällig für Cyberangriffe
sein. Eine Kompromittierung oder Manipulation von KI-Algorithmen könnte
schwerwiegende Auswirkungen auf die Energieinfrastruktur haben. Der Schutz der
KI-Systeme vor potenziellen Sicherheitsbedrohungen und die Implementierung
robuster Sicherheitsmaßnahmen sind daher von großer Bedeutung.

Zusammenfassend ist es wichtig, die Grenzen und Risiken von künstlicher
Intelligenz in Energieversorgungssystemen zu beachten. Dazu gehören
Herausforderungen bei der Datenqualität und -verfügbarkeit, ethische
Überlegungen, mögliche Fehlinterpretationen oder Fehlentscheidungen von
KI-Modellen sowie Sicherheitsrisiken. Durch eine sorgfältige Planung,
Überwachung und Implementierung können diese Risiken minimiert und die Vorteile
von KI in der Energieversorgung bestmöglich genutzt werden.
\subsection{Perspektiven}
Die Rolle von künstlicher Intelligenz und maschinellem Lernen bei der
Optimierung von Energieversorgungssystemen bietet vielversprechende
Perspektiven für die Zukunft. Das die Softwareindustrie, welche die KI-Systeme
maßgeblich mitprägt, immer schneller am wachsen ist zeigt
Abbildung(\ref{fig:wachstumsberufe}). Unter allen Berufen, ist der Informatiker
mit großem Abstand der schnellst Wachsende.
\begin{figure}[!h]
    \centering
    \includesvg[width=1\columnwidth]{bilder/Wachstumsberufe Deutschland_2.svg}
    \caption[width=0.4\textwidth]{Ablauf bei Reinforcement Learning}
    \label{fig:wachstumsberufe}
\end{figure}
\cite{eichhorst2015zukunft}

Mit dem weiteren Fortschritt von KI- und ML-Technologien eröffnen sich neue
Möglichkeiten zur Effizienzsteigerung, Kostensenkung und Verbesserung der
Nachhaltigkeit in der Energieversorgung.

Eine vielversprechende Perspektive besteht in der Integration von KI und ML in
Smart Grids und intelligenten Energienetzen. Durch den Einsatz
fortschrittlicher Algorithmen und Modelle können Energieflüsse optimiert,
Spitzenlasten reduziert und erneuerbare Energien effizienter in das
Energiesystem integriert werden. Dies ermöglicht eine bessere Ausbalancierung
von Angebot und Nachfrage und trägt zu einer stabilen und nachhaltigen
Energieversorgung bei.

Des Weiteren eröffnet die Kombination von KI und ML mit dem Internet der Dinge
(IoT) und dem Einsatz von Sensoren neue Perspektiven für die Energieeffizienz.
Intelligente Geräte und Systeme können Energieverbrauchsdaten erfassen und
analysieren, um personalisierte Empfehlungen zur Reduzierung des
Energieverbrauchs zu geben. Dies ermöglicht es Verbrauchern, bewusstere
Entscheidungen zu treffen und ihren Energieverbrauch zu optimieren.

Ein weiteres vielversprechendes Feld ist die Entwicklung autonomer
Energieversorgungssysteme. Durch den Einsatz von KI und ML können autonome
Energieerzeugungs- und Speicherlösungen entwickelt werden, die den Bedarf an
menschlichem Eingriff minimieren. Diese Systeme können den Energieverbrauch und
die -erzeugung in Echtzeit überwachen, Vorhersagen treffen und automatisch
optimale Entscheidungen treffen, um die Effizienz und Zuverlässigkeit der
Energieversorgung zu maximieren.

Darüber hinaus bieten KI und ML Möglichkeiten zur besseren Vorhersage und
Bewältigung von Energienotfällen. Durch die Analyse von Echtzeitdaten können
frühzeitige Warnungen und reaktionsschnelle Maßnahmen bei Störungen im
Energiesystem ermöglicht werden. Dies unterstützt die Sicherheit und Resilienz
der Energieinfrastruktur.

Insgesamt eröffnen die Fortschritte in KI und ML spannende Perspektiven für die
Energieversorgungssysteme der Zukunft. Mit einer kontinuierlichen Forschung und
Entwicklung, der Verfügbarkeit hochwertiger Daten und einer sorgfältigen
Integration dieser Technologien können wir eine effizientere, nachhaltigere und
zuverlässigere Energieversorgung erreichen.

\section{Fazit und Ausblick}
Die Rolle von künstlicher Intelligenz und maschinellem Lernen bei der
Optimierung von Energieversorgungssystemen ist von großer Bedeutung. Die
Anwendung von KI und ML bietet vielfältige Möglichkeiten zur
Effizienzsteigerung, Kostenoptimierung, Integration erneuerbarer Energien und
Verbesserung des Verbraucherverhaltens. Durch den Einsatz von KI können
Energieversorgungsunternehmen fundierte Entscheidungen treffen, Prognosen
erstellen, den Energiefluss optimieren und die Zuverlässigkeit der
Energieversorgung verbessern.

Jedoch sind bei der Nutzung von KI in der Energieversorgung auch
Herausforderungen zu berücksichtigen. Grenzen wie die Qualität und
Verfügbarkeit von Daten, ethische Überlegungen, potenzielle
Fehlinterpretationen oder Fehlentscheidungen von KI-Modellen sowie
Sicherheitsrisiken erfordern eine sorgfältige Planung, Überwachung und
Implementierung.

Ausblick: Für die Zukunft bieten sich auf dem Gebiet der künstlichen
Intelligenz und des maschinellen Lernens in der Energieversorgung zahlreiche
Möglichkeiten. Fortschritte in der Datenverfügbarkeit und -qualität werden die
Genauigkeit und Leistungsfähigkeit von KI-Modellen verbessern. Eine verstärkte
Berücksichtigung ethischer Prinzipien bei der Entwicklung und Implementierung
von KI-Systemen wird zu einer fairen und transparenten Nutzung beitragen.

Darüber hinaus könnten Fortschritte im Bereich des Verständnis von
KI-Modellen dazu beitragen, Vertrauen und Akzeptanz in deren Entscheidungen zu
fördern. Die Integration von KI-Systemen mit dem Internet der Dinge (IoT) und
anderen fortschrittlichen Technologien ermöglicht eine noch intelligentere und
effizientere Energieversorgung.

Ein wichtiger Aspekt für den Ausblick ist die kontinuierliche Forschung und
Entwicklung auf dem Gebiet der künstlichen Intelligenz und des maschinellen
Lernens in der Energieversorgung. Neue Algorithmen, Modelle und Techniken
werden entwickelt, um die Herausforderungen anzugehen und die
Leistungsfähigkeit von KI-Systemen weiter zu verbessern.

Insgesamt bietet die Kombination von künstlicher Intelligenz und maschinellem
Lernen enorme Potenziale für die Optimierung von Energieversorgungssystemen.
Mit einer sorgfältigen Betrachtung der Grenzen, Risiken und Chancen können wir
eine nachhaltigere, effizientere und zuverlässigere Energieversorgung der
Zukunft erreichen.

\printbibliography
\end{document}