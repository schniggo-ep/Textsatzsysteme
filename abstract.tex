Die Optimierung von Energieversorgungssystemen ist angesichts der steigenden
Nachfrage nach effizienter, zuverlässiger und nachhaltiger Energieversorgung
von großer Bedeutung. In diesem Zusammenhang spielen künstliche Intelligenz
(KI) und maschinelles Lernen (ML) eine immer wichtigere Rolle. Dieser Artikel
untersucht die Rolle von KI und ML bei der Optimierung von
Energieversorgungssystemen. Es werden verschiedene Anwendungen von KI und ML
vorgestellt, darunter Prognose und Optimierung, Lastmanagement und
Nachfragesteuerung, Anlagenüberwachung und Wartung, Integration erneuerbarer
Energien sowie Energieeffizienz und Verbraucherempfehlungen. Darüber hinaus
werden die Herausforderungen diskutiert, die mit der Optimierung von
Energieversorgungssystemen verbunden sind, wie die Komplexität der Systeme, die
Volatilität erneuerbarer Energien, die Skalierbarkeit und Flexibilität, die
Sicherheit und Zuverlässigkeit sowie die wirtschaftlichen und regulatorischen
Aspekte. Die Erkenntnisse dieser Arbeit tragen dazu bei, das Verständnis für
die Rolle von KI und ML bei der Optimierung von Energieversorgungssystemen zu
vertiefen und zeigen Potenziale sowie Herausforderungen auf, die bei der
Einführung dieser Technologien berücksichtigt werden sollten.