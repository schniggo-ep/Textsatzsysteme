Der Einsatz künstlicher Intelligenz in Energieversorgungssystemen birgt
sowohl Grenzen als auch Risiken, die berücksichtigt werden müssen. Eine der
Grenzen liegt in der Qualität und Verfügbarkeit von Daten. KI-Modelle erfordern
hochwertige und ausreichende Daten, um zuverlässige Vorhersagen und
Entscheidungen treffen zu können. In der Energieversorgung können jedoch
Engpässe bei der Datenqualität und -verfügbarkeit auftreten, insbesondere wenn
es um spezifische Daten zu erneuerbaren Energien oder Verbrauchsverhalten geht.
Eine unzureichende Datenbasis kann die Genauigkeit und Leistungsfähigkeit von
KI-Anwendungen beeinträchtigen. Zudem ist umfässt die Datenbeschaffung für
künstliche Intelligenzen das Problem der Privatsphäre. Eine KI funktioniert
besser, desto mehr Eingabedaten angeboten werden können. Jedoch birgt eine
digitale Datenerfassung, vorallem von persönlichen Daten, zum Beispiel des
häuslichen Energieverbrauchs immer das Risiko, dass diese missbräuchlich
genutzt werden können. Das würde einen genauen rechtlichen Rahmen benötigen, in
welchem Sensordaten für die KI gesammelt werden dürfen. Die einfachste Lösung
ist hierbei eine Entwicklung der KI-Systeme, welche ihre Effektivität mit
weniger Eingabedaten halten.\cite{dreyer2019kunstliche}

Des Weiteren können Sicherheitsrisiken auftreten. KI-Systeme, die mit
Energieversorgungssystemen verbunden sind, könnten anfällig für Cyberangriffe
sein. Eine Kompromittierung oder Manipulation von KI-Algorithmen könnte
schwerwiegende Auswirkungen auf die Energieinfrastruktur haben. Der Schutz der
KI-Systeme vor potenziellen Sicherheitsbedrohungen und die Implementierung
robuster Sicherheitsmaßnahmen sind daher von großer Bedeutung. Zudem ist
Energieversorgung nicht nur gefährdet bei einem generellen Angriff auf das
System. Hacker können durch gezielte Manipulationen die Entscheidungsfindung
der KI unauffällig nach ihren Vorstellungen beeinflussen.
\cite{dreyer2019kunstliche}

Ein weiterer Aspekt sind die ethischen Überlegungen. KI-Systeme können
Entscheidungen treffen, die weitreichende Auswirkungen auf die
Energieversorgung und die Gesellschaft haben. Es besteht das Risiko von
Vorurteilen, Diskriminierung oder unfairen Entscheidungen, wenn die Algorithmen
nicht angemessen trainiert oder kalibriert sind. Es ist wichtig,
sicherzustellen, dass KI-Systeme fair und transparent arbeiten, um potenzielle
negative Auswirkungen zu vermeiden. Hinter diesen Befürchtungen steht oft die
sogenannte Singularity-Debatte. Diese Debatte befasst sich mit dem Gefahr, dass
künstliche Intelligenz eines Tages zu mächtig wird. Bestärkt wird dieser
Gedanke durch einige bekannte Personen wie Elon Musk. Anhänger der Singularity
Debatte begründen ihre Sorgen häufig mit der These, KI-Systeme könnten sich in
einigen Jahren selbst weiterentwickeln.\cite{buxmann2021ethische}

Darüber hinaus besteht das Risiko von Fehlinterpretationen oder
Fehlentscheidungen durch KI-Modelle. Obwohl KI-Systeme in der Lage sind, Muster
und Zusammenhänge in den Daten zu erkennen, sind sie nicht immun gegen Fehler.
Wenn die Trainingsdaten nicht repräsentativ oder unvollständig sind, können die
Vorhersagen und Empfehlungen der KI-Systeme ungenau oder fehlerhaft sein. Eine
sorgfältige Validierung und Überwachung der KI-Modelle ist daher von
entscheidender Bedeutung, um mögliche Risiken zu minimieren.

Zusammenfassend ist es wichtig, die Grenzen und Risiken von künstlicher
Intelligenz in Energieversorgungssystemen zu beachten. Dazu gehören
Herausforderungen bei der Datenqualität und -verfügbarkeit, ethische
Überlegungen, mögliche Fehlinterpretationen oder Fehlentscheidungen von
KI-Modellen sowie Sicherheitsrisiken. Durch eine sorgfältige Planung,
Überwachung und Implementierung können diese Risiken minimiert und die Vorteile
von KI in der Energieversorgung bestmöglich genutzt werden.