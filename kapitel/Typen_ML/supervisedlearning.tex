Der Grundgedanke bei supervised Learning ist, dass die Ausgabewerte bekannt
sind. Man unterscheidet hierbei zwischen konkreten Ausgabewerten, sowie
kontinuierliche Werte. Diese beiden Wertearten bestimmen, ob es sich bei dem
supervised-learning Ansatz um eine Klassifikation oder eine Regression handelt.
Bei einem kontinuierlichen Ausgangstyp ist es eine Regression, bei einem
diskreten Ausgabewert spricht man von einer Klassifikation. Diese
Unterscheidung ist nur von den Ausgabetypen abhängig und nicht von den
Eingabewerten.\cite{matzka2021unuberwachtes}\\
