Beim unüberwachten Lernen gibt es keine Zielwerte, sondern das Ziel besteht
darin, Muster und Strukturen in den Daten zu entdecken. Für das System liegen
lediglich Eingabewerte vor, jedoch nicht die zugehörigen Ausgabewerte, wie beim
überwachten Lernen.\cite{lanquillon2019grundzuge}Das Modell lernt eigenständig,
wie es die Daten gruppieren oder Zusammenhänge finden kann. Clustering und
Dimensionsreduktion sind Beispiele für unüberwachtes Lernen. Im Folgenden wird
Clustering genauer erläutert: Beim hierarchischen Clustering geht es darum,
Entfernungen zwischen verschiedenen Punkten zu messen. Eine der einfachsten
Arten, die Distanz zu ermitteln ist die sogenannte Manhattan-Funktion. Diese
spiegelt den Weg wieder, der gegangen werden müsste, wenn man einem Gitternetz
folgt. Betrachtet man die Punkte X=(X1,X2,...) und Y=(Y1,Y2,...) ergibt sich
die folgende Formel:
\begin{equation*}
    d = \sum_{i=1}^n |X_i -Y_i|
\end{equation*}