Reinforcement learning unterscheidet sich grundlegend von den bisher genannten
Lernarten. Während bei un-, semisupervised learning, sowie supervised learning
immer sehr viele Eingabedaten für das Lernsystem bereitgestellt werden müssen,
generiert reinforcement learning diese selbst.Ein sogenannter Agent interagiert
hierfür in einer bestimmten Umgebung. Das Ziel des Agenten besteht darin,
Aktionen auszuführen, um eine maximale Belohnung zu erhalten. Durch
Erfahrungen, Belohnungen und Bestrafungen passt der Agent sein Verhalten an, um
optimale Entscheidungen in der Umgebung zu treffen. Dieser Lernansatz ist
perfekt für Szenarien, in denen man keine, oder nur sehr schwierig
Trainingsdaten bereitstellen kann. \cite{lanquillon2019grundzuge}

\begin{figure}
    \centering
    \includesvg[width=0.4\textwidth]{bilder/reinforcementlearning.svg}
    \caption[width=0.4\textwidth]{Ablauf bei Reinforcement Learning}
    \label{fig:disreinforcementlearning}
\end{figure}