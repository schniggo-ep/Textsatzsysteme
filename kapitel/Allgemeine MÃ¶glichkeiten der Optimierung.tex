Die Optimierung der Energiesysteme ist in unterschiedliche Bereiche aufteilbar.
Zum einen sollte auf die Performance der Systeme geachtet werden. Durch den
Mehrverbrauch an Energie im Zuge der Digitalisierung und des
Bevölkerungswachstums, ist es von immer größerer Bedeutung, ein sicheres,
starkes Energienetz zur Verfügung zu haben. !!!!!GRAFIK EINBINDEN
ENERGIEVERBRAUCH!!! Die Steigerung der Performance kann durch den Ausbau von
bereits existierenden Anlagen geschehen, jedoch sollte man auch versuchen, in
der Forschung neue Möglichkeiten zu finden, den Wirkungsgrad von Systemen zu
erhöhen. Dadurch könnte man dem hohen Flächenverbrauch, durch Solar- und
Windparks entgegenwirken.

Ein weiterer Bereich, auf welchen geachtet werden muss, ist die
Umweltverträglichkeit der Energieversorgungssysteme. Vor dem Hintergrund des
Klimawandels, sollte möglichst vermieden werden,dass große Mengen an $CO_2$ und
anderen Gasen, durch konventionelle Energieversorgungssysteme in die Atmosphäre
gelangen. Durch Verbesserung von regenerativen Systemen kann diesem Vorgang vorgebeugt werden.