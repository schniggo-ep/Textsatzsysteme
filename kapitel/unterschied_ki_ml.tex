Künstliche Intelligenz ist ein umfassendes Konzept, das sich auf die
Entwicklung von Computern oder Maschinen bezieht, die in der Lage sind,
Aufgaben auszuführen, die normalerweise menschliche Intelligenz erfordern.
KI-Systeme sollen menschenähnliches Denken, Lernen, Problemlösen,
Entscheidungsfindung und Sprachverarbeitung nachbilden. Sie können auf
verschiedenen Ansätzen basieren, darunter maschinelles Lernen.

Maschinelles Lernen ist ein Teilgebiet der künstlichen Intelligenz, das
sich darauf konzentriert, Algorithmen und Techniken zu entwickeln, die es
Computern ermöglichen, aus Erfahrungen zu lernen und Aufgaben zu erledigen,
ohne explizit programmiert zu sein. Im Kern geht es beim maschinellen Lernen
darum, Muster und Zusammenhänge in Daten zu erkennen und darauf basierend
Vorhersagen zu treffen oder Entscheidungen zu treffen.

Der Unterschied zwischen KI und ML besteht darin, dass KI ein übergeordnetes
Konzept ist, das sich auf den allgemeinen Bereich der Entwicklung intelligenter
Systeme bezieht, während ML eine konkrete Methode ist, die von KI-Forschern und
-Entwicklern verwendet wird, um Computermodelle zu erstellen, die auf Daten
lernen können.

Ein KI-System kann auf verschiedenen Techniken basieren, einschließlich
maschinellem Lernen, aber es kann auch andere Ansätze wie logisches
Schlussfolgern, Expertensysteme, neuronale Netzwerke oder symbolische
Verarbeitung verwenden. Das maschinelle Lernen hingegen ist eine spezifische
Methode innerhalb der KI, bei der Algorithmen verwendet werden, um aus Daten zu
lernen und Vorhersagen zu treffen, ohne dass explizite Regeln programmiert
werden müssen.

Zusammenfassend lässt sich sagen, dass KI ein übergeordnetes Konzept ist, das
sich auf die Entwicklung intelligenter Systeme bezieht, während ML eine
spezifische Methode ist, die in der KI eingesetzt wird, um Computermodelle zu
erstellen, die aus Erfahrungen lernen können. Maschinelles Lernen ist ein
wichtiger Teil der KI, aber es gibt auch andere Techniken und Ansätze, die in
KI-Systemen verwendet werden können.\cite{metaversekunstliche}\cite{kirste2019einleitung}