Die Rolle von künstlicher Intelligenz und maschinellem Lernen bei der
Optimierung von Energieversorgungssystemen bietet vielversprechende
Perspektiven für die Zukunft. Das die Softwareindustrie, welche die KI-Systeme
maßgeblich mitprägt, immer schneller am wachsen ist zeigt
Abbildung(\ref{fig:wachstumsberufe}). Unter allen Berufen, ist der Informatiker
mit großem Abstand der schnellst Wachsende.
\begin{figure}[!h]
    \centering
    \includesvg[width=1\columnwidth]{bilder/Wachstumsberufe Deutschland_2.svg}
    \caption[width=0.4\textwidth]{Ablauf bei Reinforcement Learning}
    \label{fig:wachstumsberufe}
\end{figure}
\cite{eichhorst2015zukunft}

Mit dem weiteren Fortschritt von KI- und ML-Technologien eröffnen sich neue
Möglichkeiten zur Effizienzsteigerung, Kostensenkung und Verbesserung der
Nachhaltigkeit in der Energieversorgung.

Eine vielversprechende Perspektive besteht in der Integration von KI und ML in
Smart Grids und intelligenten Energienetzen. Durch den Einsatz
fortschrittlicher Algorithmen und Modelle können Energieflüsse optimiert,
Spitzenlasten reduziert und erneuerbare Energien effizienter in das
Energiesystem integriert werden. Dies ermöglicht eine bessere Ausbalancierung
von Angebot und Nachfrage und trägt zu einer stabilen und nachhaltigen
Energieversorgung bei.

Des Weiteren eröffnet die Kombination von KI und ML mit dem Internet der Dinge
(IoT) und dem Einsatz von Sensoren neue Perspektiven für die Energieeffizienz.
Intelligente Geräte und Systeme können Energieverbrauchsdaten erfassen und
analysieren, um personalisierte Empfehlungen zur Reduzierung des
Energieverbrauchs zu geben. Dies ermöglicht es Verbrauchern, bewusstere
Entscheidungen zu treffen und ihren Energieverbrauch zu optimieren.

Ein weiteres vielversprechendes Feld ist die Entwicklung autonomer
Energieversorgungssysteme. Durch den Einsatz von KI und ML können autonome
Energieerzeugungs- und Speicherlösungen entwickelt werden, die den Bedarf an
menschlichem Eingriff minimieren. Diese Systeme können den Energieverbrauch und
die -erzeugung in Echtzeit überwachen, Vorhersagen treffen und automatisch
optimale Entscheidungen treffen, um die Effizienz und Zuverlässigkeit der
Energieversorgung zu maximieren.

Darüber hinaus bieten KI und ML Möglichkeiten zur besseren Vorhersage und
Bewältigung von Energienotfällen. Durch die Analyse von Echtzeitdaten können
frühzeitige Warnungen und reaktionsschnelle Maßnahmen bei Störungen im
Energiesystem ermöglicht werden. Dies unterstützt die Sicherheit und Resilienz
der Energieinfrastruktur.

Insgesamt eröffnen die Fortschritte in KI und ML spannende Perspektiven für die
Energieversorgungssysteme der Zukunft. Mit einer kontinuierlichen Forschung und
Entwicklung, der Verfügbarkeit hochwertiger Daten und einer sorgfältigen
Integration dieser Technologien können wir eine effizientere, nachhaltigere und
zuverlässigere Energieversorgung erreichen.