Künstliche neuronale Netze sind die Grundfunktionsweise von System, welche
künstliche Intelligenz nutzen. Diese Netze bestehen aus einer großen Menge an
Neuronen, welche, genauso wie die Neuronen im menschlichen Gehirn, eine
Vielzahl an Daten verarbeitet. In der folgenden Abbildung ist der Ablauf der
Arbeitsweise von künstlichen neuronalen Netzen exemplarisch dargestellt:

\begin{figure}[h]
    \centering
    \includesvg[width=1\columnwidth]{bilder/neuronalenetze_prinzip.svg}
    \caption[width=0.4\textwidth]{Ablauf des Trainings eines neuronalen Netzes}
    \label{fig:trainingneuronalesnetz}
\end{figure}

Das Training eines neuronalen Netzes beginnt mit dem Bereitstellen der
Informationen, welches oft Signale oder Bildpixel sind. Zum Beispiel
Handschriften, Gesichter oder Daten von Wetterberichten. Darauffolgend wird der
Aufbau des neuronalen Netzes bestimmt, hinsichtlich der Anzahl an Hidden-Layers
und Neuronen, gefolgt von der Verarbeitung der Eingangsinformationen durch
Training mit verschiedenen Lernverfahren. Das Lernverfahren, mit dem geringsten
Fehler wird ausgewählt, aus welchem dann die Antwort des neuronalen Netzes in,
als Vektor kodierter Form ausgegeben wird.\cite{dorn2016programmieren}