Dezentrale Energieversorgungssysteme zeichnen sich durch eine lokalisierte
Energieerzeugung und -verteilung aus. Anstatt auf zentrale Kraftwerke
angewiesen zu sein, werden kleine Energieerzeugungsanlagen wie Solaranlagen auf
Dächern von Gebäuden oder kleine Windturbinen eingesetzt. Diese Systeme bieten
die Möglichkeit, Energie vor Ort zu erzeugen und den Energieverlust durch den
Transport über weite Strecken zu reduzieren.