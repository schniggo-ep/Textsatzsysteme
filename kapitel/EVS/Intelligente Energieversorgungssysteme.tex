Intelligente Energieversorgungssysteme nutzen fortschrittliche Technologien wie
intelligente Netze (Smart Grids), um die Effizienz, Zuverlässigkeit und
Integration erneuerbarer Energien zu verbessern. Smart Grids ermöglichen eine
bidirektionale Kommunikation zwischen Energieerzeugern, Verbrauchern und dem
Netz, wodurch ein optimierter Energiefluss, eine bessere Laststeuerung und die
Integration von dezentralen Erzeugungsanlagen möglich sind. Durch den Einsatz
von Sensoren, Datenanalyse und Steuerungssystemen können intelligente
Energieversorgungssysteme den Energieverbrauch und die Nachfrage besser
prognostizieren und optimieren.