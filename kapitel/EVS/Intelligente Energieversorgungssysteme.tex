Da in den vergangenen Jahrzehnten die Anzahl an erneuerbaren
Energieversorgungssystemen extrem gestiegen ist, und auch in Zukunft aufgrund
des Klimwandels immer mehr werden wird, werden Möglichkeiten benötigt, die
Energie bestens zu verteilen. Eine Herausforderung hierbei ist die Inkontinenz
von Energieerzeugung von z.B. Windkraftanlagen begründet durch die schwankende
Windstärke. Intelligente Energieversorgungssysteme nutzen fortschrittliche
Technologien wie intelligente Netze (Smart Grids), um die Effizienz,
Zuverlässigkeit und Integration erneuerbarer Energien zu verbessern. Smart
Grids ermöglichen eine bidirektionale Kommunikation zwischen Energieerzeugern,
Verbrauchern und dem Netz, wodurch ein optimierter Energiefluss, eine bessere
Laststeuerung und die Integration von dezentralen Erzeugungsanlagen möglich
sind.\cite{lund2012electricity} Durch den Einsatz von Sensoren, Datenanalyse und Steuerungssystemen
können intelligente Energieversorgungssysteme den Energieverbrauch und die
Nachfrage besser prognostizieren und optimieren.