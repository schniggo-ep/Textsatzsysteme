Vorallem da konventionelle Energieversorgungssysteme meist auf fossilen
Brennstoffen basieren, welche irgendwann ausgeschöpft sein werden, stellt sich
die Frage nach neuen Energiequellen, welche sich selbst regenerieren.
Erneuerbare Energieversorgungssysteme nutzen natürliche Ressourcen wie Sonne,
Wind, Wasser und Biomasse zur Energieerzeugung. Solarenergie wird durch
Photovoltaik- oder Solarthermieanlagen gewonnen. Photovoltaikanlagen sind
bereits Teil, der meisten Häuser, bei welchen Elektronen durch die Absorption
von Photonen in den Halbleitern der Solarzelle frei werden. Die freie, positiv
geladene Stelle, welche durch das verlassene Elektron entsteht, füllt sich
durch ein neues Elektron, welches dort wieder ein positiv geladenes Loch
verursacht. Dieser Prozess wiederholt sich dauerhaft bei Sonneneinstrahlung.
Verhindert man diese Entstehung von Elektronen-Loch-Paaren, entsteht eine
Spannung. Diese entstandene Spannung kann durch den richtigen Anschluss zum
Betreiben elektrischer Geräte mit Strom verwendet
werden.\cite{renner2010grundlagen} Solarenergie kennt man aus dem Alltag, wenn
sich Gegenstände, zum Beispiel ein Auto durch Sonneneinstrahlung aufwärmt. Die
Sonnenenergie wird in Form von Wärme im Auto durch die Isolationswirkung von
Glas, Blech, etc. gespeichert. In sogenannten Solarkollektoren ist dieser
Effekt perfektioniert umgesetzt. Die äußere Hülle bildet eine Schicht
Spezialglas, welche darauf ausgelegt ist, möglichst viel Sonnenstrahlen
durchzulassen. Darunter befindet sich ein Absorber, welcher aus einem
speziellen Material gefertigt ist, um möglichst viel der Strahlen in thermisch
hochqualitative Energie umzuwandeln.\cite{schabbach2014solarthermie} Eine
weiterere Naturkraft, welcher zur Energieerzeugung genutzt wird ist die
Windkraft. Die Umsetzung erfolgt in Windrädern, welche sehr einfach aufgebaut
sind. Der Wind treibt einen Rotor an, welcher seine Rotationsenergie an einen
Generator abgibt, welcher die kinetische Energie dann in verwendbaren Strom
umwandelt. Da sich der Wind drehen kann sind die Rotoren auf den Windrädern
gelagert, sodass diese immer optimal zum Wind stehen und so möglichst viel
kinetische Energie umwandeln
können.\cite{osterhage2015windkraft}Wasserkraftwerke nutzen die Energie des
fließenden Wassers. Meist werden hierzu Wehre eingesetzt, welche zwischen
beiden Seiten einen Höhenunterschied des Wasserspiegels aufweisen. Dieser
Höhenunterschied, wird in Kombination mit der Masse des Wassers genutzt um
Turbinen im Wehr anzutreiben. Folgende Gleichung beschreibt hier den
Zusammenhang der Kräfte, wobei $E_{Pot}$ die potentielle Energie des Wassers
beschreibt, $m_{Wa}$ die Masse des Wassers, g die Gravitationskraft und
$h_{OW}$ den Höhenunterschied der zwei Wasserspiegel.

\begin{equation*}
    E_{Pot} = m_{Wa} * g * h_{OW}
\end{equation*}

Natürlich ist das nur eine vereinfachte Form der Gleichung. Es spielen noch
viele weitere Faktoren eine Rolle, wie die Dichte des Wassers, der
Reibkoeffizient des Rohrs oder das Lager der Turbine. All diese Faktoren können
sich auf den Wirkungsgrad des Wasserkraftwerkes
auswirken.\cite{aufleger2020stromerzeugung}

Biomasseanlagen wandeln organische Materialien in Energie um. Hierbei gibts es
sehr unterschiedliche Wirkungsgrade, je nach Pflanzenart, welche verwendet
wird.
\begin{tabular}[h]
    \caption{Wirkungsgrade bei der Produktion von Biomasse}
    Ozeane           & 0,07\% & Wälder      & 0,55\% \\
    Süßwasser        & 0,50\% & Mais        & 3,2\%  \\
    Kulturlandschaft & 0,30\% & Zuckerrohr  & 4,8\%  \\
    Grasland         & 0,30\% & Zuckerrüben & 5,4\%
    \label{tab:tabelle1}
\end{tabular}

Diese Systeme sind nachhaltiger und umweltfreundlicher, da sie erneuerbare
Ressourcen nutzen und weniger CO2-Emissionen verursachen.