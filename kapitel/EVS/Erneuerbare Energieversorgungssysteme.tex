Erneuerbare Energieversorgungssysteme nutzen natürliche Ressourcen wie Sonne,
Wind, Wasser und Biomasse zur Energieerzeugung. Solarenergie wird durch
Photovoltaik- oder Solarthermieanlagen gewonnen, während Windenergie mithilfe
von Windturbinen erzeugt wird. Wasserkraftwerke nutzen die Energie des
fließenden Wassers, und Biomasseanlagen wandeln organische Materialien in
Energie um. Diese Systeme sind nachhaltiger und umweltfreundlicher, da sie
erneuerbare Ressourcen nutzen und weniger CO2-Emissionen verursachen.