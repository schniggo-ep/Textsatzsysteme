Die Zielsetzung dieser Arbeit besteht darin, die Rolle von künstlicher
Intelligenz und maschinellem Lernen bei der Optimierung von
Energieversorgungssystemen zu untersuchen. Es wird angestrebt, ein umfassendes
Verständnis dafür zu entwickeln, wie KI und ML eingesetzt werden können, um die
Effizienz, Zuverlässigkeit und Nachhaltigkeit der Energieversorgung zu
verbessern. Dabei sollen die verschiedenen Anwendungen und Herausforderungen
von KI und ML in diesem Kontext beleuchtet werden. Des Weiteren werden mögliche
Perspektiven und zukünftige Entwicklungen aufgezeigt, um einen Ausblick auf das
Potenzial dieser Technologien für die Energieversorgungssysteme der Zukunft zu
geben. Letztendlich zielt die Arbeit darauf ab, einen Beitrag zur aktuellen
Diskussion über die Integration von KI und ML in den Bereich der
Energieversorgung zu leisten und Erkenntnisse für Entscheidungsträger, Forscher
und Praktiker bereitzustellen.