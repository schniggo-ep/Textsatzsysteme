Die Energieversorgungssysteme stehen heutzutage vor einer Vielzahl von
Herausforderungen. Der steigende Energiebedarf, die begrenzte Verfügbarkeit
konventioneller Energieressourcen und die zunehmende Bedeutung erneuerbarer
Energien erfordern innovative Lösungen, um die Effizienz und Nachhaltigkeit der
Energieversorgung zu verbessern. In diesem Zusammenhang spielen künstliche
Intelligenz (KI) und maschinelles Lernen (ML) eine immer wichtigere Rolle.

Die Grundidee hinter dem Einsatz von KI und ML in der Optimierung von
Energieversorgungssystemen besteht darin, dass diese Technologien in der Lage
sind, große Mengen an Daten zu analysieren, Muster zu erkennen, Vorhersagen und adaptive Entscheidungen zu treffen. Dadurch können
Energieversorgungssysteme effizienter gestaltet, die Zuverlässigkeit erhöht und
Kosten gesenkt werden.

Ein zentrales Anwendungsgebiet von KI und ML liegt in der Prognose von
Energieerzeugung und -nachfrage. Durch den Einsatz von ML-Algorithmen können
präzisere Vorhersagen über den zukünftigen Energiebedarf gemacht werden, was
eine optimierte Planung der Energieerzeugung und der Verteilung ermöglicht.
Darüber hinaus können ML-Modelle genutzt werden, um Lastmanagement-Strategien
zu entwickeln und den Energieverbrauch zu optimieren und die Spitzenlasten zu
reduzieren.

Ein weiteres wichtiges Einsatzgebiet von KI und ML liegt in der Optimierung von
Stromnetzen und -verteilung. Hier können intelligente Algorithmen eingesetzt
werden, um den Energiefluss zu steuern, Engpässe zu identifizieren und die
Netzstabilität zu gewährleisten. ML kann auch bei der Erkennung von Fehlern und
Störungen in Energieversorgungssystemen helfen, indem Anomalien in den Daten
erkannt und potenzielle Probleme frühzeitig gemeldet werden.

Die Integration erneuerbarer Energien stellt ebenfalls eine große
Herausforderung dar, da sie von Natur aus volatil und schwer vorhersehbar sind.
Hier kommen KI und ML zum Einsatz, um Vorhersagemodelle für die Stromerzeugung
aus erneuerbaren Quellen zu entwickeln, um die Integration in das Gesamtsystem
zu optimieren. Dies ermöglicht eine bessere Ausnutzung erneuerbarer Energien
und eine verbesserte Integration von konventionellen Energiequellen.

Neben den technischen Aspekten gibt es auch weitere wichtige Faktoren, die bei
der Optimierung von Energieversorgungssystemen berücksichtigt werden müssen.
Datenschutz und Sicherheit spielen eine entscheidende Rolle, da die
Verarbeitung großer Mengen an Energie- und Verbrauchsdaten eine Herausforderung
für den Schutz sensibler Informationen darstellt. Darüber hinaus stellen sich
auch ethische Fragen, wie beispielsweise der faire Zugang zu Energie oder die
Vermeidung von Diskriminierung bei der Nutzung von KI-gesteuerten
Energiesystemen.

In Anbetracht all dieser Aspekte ist es von großer Bedeutung, die Rolle von KI
und ML bei der Optimierung von Energieversorgungssystemen zu untersuchen. Durch
die Nutzung des Potenzials dieser Technologien können wir eine effizientere,
nachhaltigere und zuverlässigere Energieversorgung erreichen. In dieser
Seminararbeit werden die verschiedenen Einsatzbereiche von KI und ML in der
Energieversorgung untersuchen und ihre Auswirkungen sowie mögliche
Herausforderungen und zukünftige Entwicklungen analysieren.