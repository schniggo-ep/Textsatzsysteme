Hintergrund dieser Arbeit ist die in letzter Zeit extrem gestiegene Bedeutung von Energieversorgung im 21. Jahrhundert, unter anderem aus dem Grund des Umweltschutzes heraus.
Man denke hier vorallem an den Umschwung mit Solaranlagen, welcher vor einigen Jahren passierte.
Vor diesem Hintergrund steht man jetzt vor der Herausforderung, solche Systeme immer effizienter zu gestalten.
Parallel dazu haben Entwickler in den vergangenen Jahren enorme Fortschritte im Thema künstlicher Intelligenz gemacht.
Seit Chat GPT hat jetzt auch die Bevölkerung Zugriff auf eine künstliche Intelligenz und gewöhnt sich daher immer mehr an die Zukunft des Machine Learnings.
Künstliche Intelligenz und Machine Learning haben ein enormes Potenzial, Effizienz, Flexibilität und die Zuverlässigkeit der Energieversorgungssystemen zu verbessern.
Jedoch gibt es auch Risiken und Herausforderungen in der Zusammenarbeit mit künstlicher Intelligenz im Hinblick auf einen Einsatz mit Energieversorgungssystemen.
Die vorliegende Arbeit zielt darauf ab, die Potenziale der Optimierung, aber auch die Gefahren zu untersuchen und zu analysieren.